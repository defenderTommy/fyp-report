%%%%%%%%%%%%%%%%%%%%%%preface.tex%%%%%%%%%%%%%%%%%%%%%%%%%%%%%%%%%%%%%%%%%
% sample preface
%
% Use this file as a template for your own input.
%
%%%%%%%%%%%%%%%%%%%%%%%% Springer %%%%%%%%%%%%%%%%%%%%%%%%%%

\preface

%% Please write your preface here
Virtual Reality is a rapidly growing field with innovative empirical applications. Two main forms of VR content can be identified in current applications: Computer generated 3D model environments and adaptations of real world scenarios through means of digital imagery and video. While both allow the user to be immersed in the VR experience, Computer Generated models are less realistic and have a high technical and financial barrier for non-technical users. On the other hand, the real world adaptations are immensely restrictive when it comes to interaction. These restrictions are based on either large storage needs or high computational needs to generate dynamic intermediate views associated with digital images. Current research literature does not provide a solution where real world simulation is achieved in immersive and interactive VR environments.

The research project is intended to provide a high degree of interaction to users who engage in immersive VR experiences related to real world adaptations. Our solution is a navigable grid map of spherical panoramas. Upon this, a threefold approach is executed as follows. Firstly, intermediate views between spheres are approximated using novel mechanisms. Secondly, an optimization strategy is introduced based on visual locality, where the areas with a higher probability of immediate interaction is given more prominence in quality while streaming. Thirdly, smooth segue transition is achieved through a machine learning backed gesture input system. Based on an extensive training dataset, the transition gestures are distinguished among various other movements associated with the VR experience. The three approaches when combined together allows for an intuitive virtual experience, while ensuring optimal resource utilization. 

\vspace{\baselineskip}
\begin{flushright}\noindent
Colombo,\hfill {\it Vipula  Dissanayake}\\
September 2015\hfill {\it Sachini  Herath}\\
\hfill {\it Sanka Rasnayaka}\\
\hfill {\it Sachith Seneviratne}\\
\hfill {\it Rajith Vidanaarachchi}\\
\end{flushright}


